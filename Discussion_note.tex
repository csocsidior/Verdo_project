\input{Setup_files/preamble}

\begin{document}
\frontmatter
\makeevenfoot{AAU}{\thepage}{}{}
\makeoddfoot{AAU}{}{}{\thepage}

\pagenumbering{gobble}
\section{EPANET model - Samlet\_slutmodel\_juni\_2013.net}

All the unconnected parts in the EPANET model are listed in the table below. The program which finds the unconnected parts was tested on Samlet\_slutmodel\_juni\_2013.net.

\begin{center}
\label{mistake_table}
    \begin{tabular}{ | p{3cm} | p{3cm} | p{3cm} | p{3cm} |}
    \hline
    \textbf{} & \textbf{Node IDs} & \textbf{Link IDs}  & \textbf{Description of the problem}  \\ 
    \hline
    1 & 20102476, ...& NY1037, ... & Tjærby is not connected to the main pipeline. Therefore, Bunkedal Vandværk does not operate. \\ 
    \hline
    2 &533, 1999 & 12757 & A pipeline left in the model, probably by accident. The presence of this does not affect anything.  \\ 
    \hline
    3 & 20102158,
    20102159& NY328 & In the low zone, a nodes and links are overlapping but not connected. \\ 
    \hline
    4 & 20102163,
    20102164 & NY331 & Same error as for Nr. 2. An unconnected pipeline under the connected links. The presence of this does not cause any error. \\ 
    \hline
    5 & 20102323, 
    2010232  & NY624 & A single unconnected pipeline. \\ 
    \hline
    6 & 20102490,
    20102491,
    20102492 & NY1065, NY1074 & A small unconnected part.  \\ 
    \hline
    \end{tabular}
    %\captionof{table}{Number of WSS elements.}
\end{center}



























% Let us write the output and constraint equations in discrete time. The output $\bar{p}_{\mathcal{K},k}$ is given by \eqref{eq1}.

% \begin{equation}
%   \label{eq1}
%   \bar{p}_{\mathcal{K},k} = K^T \bar{H}^{-T}_{\mathcal{T}}f_{\mathcal{T}}(A_2 q_{\mathcal{C},k} + A_3 K \bar{d}_{\mathcal{K},k} - A_3 D v_{\mathcal{D}} \sigma_k) - K^T\bar{H}^{-T}_{\mathcal{T}}\hat{H}^{T}_{\mathcal{T}} (\hat{p}_k + \hat{h}) - \bar{h}_{\mathcal{K}}.
% \end{equation}


% The constraint for the system is on $q_\mathcal{C}$

%  \begin{equation}
% \label{eq2}
% f_{\mathcal{C}}(q_{\mathcal{C},k}) - A_1(\hat{p}_k + \hat{h}) + A_2^T f_{\mathcal{T}}(A_2 q_{\mathcal{C},k} + A_3 K \bar{d}_{\mathcal{K},k} - A_3 D v_{\mathcal{D}} \sigma_k) = 0.
% \end{equation} 

% Reformulating \eqref{eq2} yields 

%  \begin{equation}
% \label{eq3}
% q_{\mathcal{C},k} = q_{\mathcal{C},k}\big ((\hat{p}_k + \hat{h}),\bar{d}_{\mathcal{K},k}, \sigma_k \big) .
% \end{equation} 

% Substituting \eqref{eq3} into \eqref{eq1}, we get the following


% \begin{align}
%   \label{eq4}
%       \bar{p}_{\mathcal{K},k} + \bar{h}_{\mathcal{K}} &= \nonumber K^T \bar{H}^{-T}_{\mathcal{T}}f_{\mathcal{T}}(A_2 q_{\mathcal{C},k}\big ((\hat{p}_k + \hat{h}),\bar{d}_{\mathcal{K},k}, \sigma_k \big) + A_3 K \bar{d}_{\mathcal{K},k} - A_3 D v_{\mathcal{D}} \sigma_k)   \\ &  - K^T\bar{H}^{-T}_{\mathcal{T}}\hat{H}^{T}_{\mathcal{T}} (\hat{p}_k + \hat{h}) .
% \end{align}

% In \eqref{eq4}, we have the elevation of the pumping stations $(\bar{p}_{\mathcal{K},k} + \bar{h}_{\mathcal{K}})$ given by the expression on the right-hand side. Let us write \eqref{eq4} in a form where the expression is replaced with a non-linear function $\tilde{f}_1$ with an unknown structure but with the same variables in the argument.

% \begin{equation}
%   \label{eq5}
%      \tilde{y}_k = \bar{p}_{\mathcal{K},k} + \bar{h}_{\mathcal{K}} = \tilde{f}_1 \big((\hat{p}_k + \hat{h}),\bar{d}_{\mathcal{K},k}, \sigma_k\big).
% \end{equation}

% \subsection{State equation}

% The state equation in continuous form is given by \eqref{eq6}.

% \begin{equation}
% \label{eq6}
% \Lambda \dot{\hat{p}} = - (\hat{H}_{\mathcal{C}} - \hat{H}_{\mathcal{T}} \bar{H}^{-1}_{\mathcal{T}}\bar{H}_{\mathcal{C}})  q_\mathcal{C}  - \hat{H}_{\mathcal{T}} \bar{H}^{-1}_{\mathcal{T}} K \bar{d}_{\mathcal{K}} + \hat{H}_{\mathcal{T}} \bar{H}^{-1}_{\mathcal{T}} D v_{\mathcal{D}} \sigma .
% \end{equation}

% After substituting the constraint on $q_{\mathcal{C},k}$, the discrete form, using Euler-method, is given by \eqref{eq7}.

% \begin{align}
% \label{eq7}
% \Lambda\nonumber \frac{1}{T_s} (\hat{p}_{k+1} - \hat{p}_k)  &= - (\hat{H}_{\mathcal{C}} - \hat{H}_{\mathcal{T}} \bar{H}^{-1}_{\mathcal{T}}\bar{H}_{\mathcal{C}})  q_{\mathcal{C}}\big ((\hat{p}_k + \hat{h}),\bar{d}_{\mathcal{K},k}, \sigma_k \big) \\ & - \hat{H}_{\mathcal{T}} \bar{H}^{-1}_{\mathcal{T}} K \bar{d}_{\mathcal{K},k} + \hat{H}_{\mathcal{T}} \bar{H}^{-1}_{\mathcal{T}} D v_{\mathcal{D}} \sigma_k.
% \end{align}

% \underline{Original idea:}

% The original idea was to choose a structure for \eqref{eq7} such that 

% \begin{equation}
% \label{eq8}
% \hat{p}_{k+1} = \tilde{f}_2\big( (\hat{p}_k + \hat{h}),\bar{d}_{\mathcal{K},k}, \sigma_k\big) + a_2 \hat{p}_k,
% \end{equation}

% where $a$ is a linear parameter.

% \newpage

% \underline{Other idea:} 

% \eqref{eq7} can be written in the form

% \begin{align}
% \label{eq9}
% \nonumber  \hat{p}_{k+1} - \hat{p}_k  &= T_s \Lambda^{-1} \big[- (\hat{H}_{\mathcal{C}} - \hat{H}_{\mathcal{T}} \bar{H}^{-1}_{\mathcal{T}}\bar{H}_{\mathcal{C}})  q_{\mathcal{C},k}\big ((\hat{p}_k + \hat{h}),\bar{d}_{\mathcal{K},k}, \sigma_k \big) \\ & - \hat{H}_{\mathcal{T}} \bar{H}^{-1}_{\mathcal{T}} K \bar{d}_{\mathcal{K},k} + \hat{H}_{\mathcal{T}} \bar{H}^{-1}_{\mathcal{T}} D v_{\mathcal{D}} \sigma_k \big].
% \end{align}

% In this arrangement, let us keep the tank pressures $\hat{p}_k$ on the left hand-side. Thus, $\hat{p}_{k+1}$ can be reconstructed by adding the current values $\hat{p}_k$ to both sides in the equation. Thereby, we do not need to introduce the linear parameter when we identify the state equation.  

% ($\Lambda$ is invertible, as it is a diagonal matrix, with positive values in each diagonal entry.)

% \underline{+1 on the state model:} 

% Let us recall \eqref{eq8}.

% \begin{equation}
% \label{eq10}
% \hat{p}_{k+1} = \tilde{f}_2\big( (\hat{p}_k + \hat{h}),\bar{d}_{\mathcal{K},k}, \sigma_k\big) + a_2 \hat{p}_k,
% \end{equation}

% In this equation, the present values of the states are already represented inside the non-linear term $\tilde{f}_2$ as $(\hat{p}_k + \hat{h}_k)$. $\hat{p}$ and $(\hat{p}_k + \hat{h}_k)$ are dependant variables, and we did not introduce linear terms for variables that are dependant e.g. in the output equation in \eqref{eq5}. 

\end{document}