\chapter{State of the art system identification analysis}
\label{identification_methods}

\emph{In this chapter, first an overview is given about the different methods for non-linear system identification. Secondly, the chosen method for identification is discussed in detail putting focus on the structure of the network.}

The data available in the simulation framework allows us to make a system model with the proposed control structure, however the complexity would be high. Try some black-box identification which can reproduce the same properties of the original simulation network but easier to handle in control. 

\textbf{Comments for supervisors:}
\newline
\textcolor{blue}{The data that should be gathered for any kind of system identification: 
\newline
$\sigma$ - the overall consumption in the system in different time steps. For example for a 24 hours time period.
\newline 
$\bar{p}_{\mathcal{D}}$ - pressure measurements in different non-inlet points. 
\newline
$\hat{d}_{\mathcal{K}}$ - information about how the two main pumps actuated to get the output data
\newline
Along with this, in case of a neural-based network approach how the structure of the control model should be incorporated in the identification. Because in e grey-scale system identification approach, the information about the structure derived in the system model is considered. }