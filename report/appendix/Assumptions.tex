\chapter{Assumption List}
\label{assumptionlist}

\emph{In this part of the appendix, the different assumptions and simplifications applied in the project, are collected. In order to ease the reading, each assumption has a section reference, which points back to the place where it was used.}

\begin{center}
\begin{tabular}{| >{\centering\arraybackslash}m{1in} | >{\centering\arraybackslash}m{3in} | >{\centering\arraybackslash}m{1in} | >{\centering\arraybackslash}m{1in} |}
\hline
\textbf{Number} & \textbf{Assumptions} & \textbf{Section reference} \\
\hline
\multirow{1}{1em}{1}
& The fluid in the network is water. & \secref{hydraulic_head} \\ 
\hline
\multirow{1}{1em}{2} 
& All pipes in the system are filled up fully with water at all time. & \secref{pipe_component} \\ 
\hline
\multirow{1}{1em}{3} 
& The pipes have a cylindrical structure and the cross section, $A(x)$, is constant for every $x \in [0,L]$.  & \secref{pipe_component} \\ 
\hline
\multirow{1}{1em}{4} 
& The flow of water is uniformly distributed along the cross sectional area of the pipe and the flow is turbulent. & \secref{pipe_component} \\ 
\hline
\multirow{1}{1em}{5} 
& At high flows, the Reynolds number is assumed to be constant. Therefore the Darcy friction factor, $f_D$ is assumed to be constant. & \secref{pipe_component} \\ 
\hline
\multirow{1}{1em}{6} 
& Pumps in the network are of the type centrifugal. & \secref{pump_component} \\ 
\hline
\multirow{1}{1em}{7} 
& Tanks in the network have constant diameters. Equivalently, walls of the tanks are vertical. & \secref{elevatedreservoir_component} \\ 
\hline
\multirow{1}{1em}{8} 
& Functions describing the pressure drops regarding the flow through them across the components of the system are continuously differentiable. & \secref{multi_inlet_reduced_network_description} \\ 
\hline
\multirow{1}{1em}{9} 
& The demand distribution parameter of the Multi-inlet, multi-WT model $v_{\mathcal{D}}$ is assumed to be constant in the system identification. & \secref{model_structure_of_the_multi_inlet_multi_WT_system} \\ 
\hline



\end{tabular}
\captionof{table}{List of assumptions}
\end{center}