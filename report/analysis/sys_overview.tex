\emph{This chapter gives a general overview of hydraulic systems and an introduction to the WSS in Randers. The basic topology and structures of water supply networks are explained. }

\section{Hydraulic system overview}
\label{hydraulic_system_overview}

WSSs are designed to deliver water to consumers in terms of sufficient pressure and appropriate chemical composition. Distribution systems as such are typically transport water from one geographical place to another. In practice, there are different methods existing to achieve this water transport. One example is to make use of natural advantages such as the water stored in mountains, and therefore use the potential energy to provide pressure in the network. Good examples are countries like Norway and Sweden where the advantages of the landscape can be exploited. However, in this project the source of the water is considered as groundwater, considering that in Denmark all reservoirs in the network are tapping water from the ground. After tapping the water, it goes through a cleaning process at the waterwork and afterwards the pure water is pumped into the network \cite{prahata}. In WSSs, pumps and valves are the elements that enable the delivery of water to the consumers or to elevated reservoirs, storing water for later use. Such a network is illustrated in the figure below: 

%illustration of WSS
\begin{figure}[H]
\centering
\includegraphics[width=0.35\textwidth]{report/pictures/missingfigure}
\caption{Illustration of a WSS.}
\label{fig:WSS_example}
\end{figure}

The delivered water needs to fulfil a certain pressure criteria in order to reach consumers at higher levels. For example, in some cases the pressure has to be high enough to make it to the fourth floor of a building and still provide appropriate pressure in the water taps. In such cases, generally booster pumps are placed in the area, helping to supply pressure. Too large pressure values however  increase water losses due to pipe waste \cite{walski2003advanced}.

Another criteria is that the flows through particular pipes need to stay within acceptable limits. A low flow rate can lead to water quality problems due to the undesirable microorganisms in the water and due to the metal and salt accumulation on the wall of the pipes \cite{walski2003advanced}. 

As can be seen in \figref{fig:WSS_example}, typically WSSs consist of pipe, valve, reservoir, elevated reservoir(tank) and pump components. The common property of these components is that they are all two-terminal components, therefore they can be characterized by the dynamic relationship between the pressure drop across the two endpoints and the flow through the elements. 

\subsection{Pipe networks}
\label{pipe_networks}

Pipes are the most common components of WSSs since they are used for carrying pressurized water. They serve as a connection between components. The network can be separated into different subsections, concerning the physical size and purpose of the pipes. Water supply networks consist of transmission mains, arterial mains, distribution mains and service lines as shown in the example below: 

%illustration of pipe topology
\begin{figure}[H]
\centering
\includegraphics[width=0.35\textwidth]{report/pictures/missingfigure}
\caption{Illustration of pipe mains.}
\label{fig:pipemain_example}
\end{figure}

Transmission mains deliver large amounts of water over long distances. Arterial and distribution mains provide intermediate steps towards delivering water to the end-users. Service lines transmit the water from the distribution mains straight to the end-users \cite{grigg2012water}.


\section{The Randers water supply network}
\label{the_randers_water_supply_network}