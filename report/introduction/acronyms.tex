\chapter{Nomenclature}

\section*{Acronyms}
	
	\begin{tabular}{|l c l|} \hline
		ANN     &&	Artificial Neural Network						\\ \hline
		CT      &&	Circuit Theory									\\ \hline
		D-W     &&	Darcy-Weisbach									\\ \hline
		EPA     &&	Environmental Protection Agency					\\ \hline
		FCV     &&	Flow Control Valve								\\ \hline
		GIS     &&	Geographic Information Systems					\\ \hline
		GT      &&	Graph Theory									\\ \hline
		KCL     &&	Kirchhoff's current law							\\ \hline
		LS      &&	Least squares									\\ \hline
		MPC 	&&	Model Predictive Control						\\ \hline
		NN  	&&	Neural Network									\\ \hline		
		PCA     &&	Principal Component Analysis					\\ \hline
		PRV     &&	Pressure Regulating Valve						\\ \hline
		RBF     &&	Radial Basis Function							\\ \hline
		RBFNN   &&	Radial Basis Function Neural Network			\\ \hline
		SCADA   &&	Supervisory Control And Data Acquisition		\\ \hline
		SS      &&	State-Space										\\ \hline
		TS      &&	Training Set									\\ \hline
		OD		&&	Opening Degree									\\ \hline
		ODE		&&	Ordinary Differential Equation					\\ \hline
		WSS 	&&	Water Supply System								\\ \hline
		WT  	&&	Water Tank										\\ \hline
	\end{tabular}

\section*{Acronyms - Randers Network}
	
	\begin{tabular}{|l c l|} \hline
		BKV     &&	Bunkedal Water Work 									\\ \hline
		LZ      &&	Low Zone												\\ \hline
		HBP     &&	Hobrovej Pumping Station 								\\ \hline
		HNP     &&	Hornbæk Pumping Station 								\\ \hline
		HSP 	&&	Hadsundvej Pumping Station 								\\ \hline		
		HZ      &&	High Zone												\\ \hline
		OMV		&&	Oust Mølle Water Work 									\\ \hline
		TA 		&&	Water tank A in Hobrovej 								\\ \hline
		TB 		&&	Water tank B in Hobrovej 								\\ \hline
		TC 		&&	Water tank C in Hadsundvej 								\\ \hline
		TBP 	&&	Toldbodgade Pumping Station 							\\ \hline
		VSV 	&&	Vilstrup Water Work 									\\ \hline
		ØSV 	&&	Østrup Skov Water Work 									\\ \hline
	\end{tabular}

\newpage

\section*{Symbols}


\begin{tabular}{l l l} 
	\textbf{Symbol}		&	\textbf{Description}								& \hspace{25mm}\textbf{Unit}	\\\hline

	$A_{wt}$ 			&	Cross sectional area of water tanks	 				& \hspace{25mm}[$m^2$]\\
	$a_{h2},a_{h1},a_{h0}$ &	Centrifugal Pump constants	 								& \hspace{25mm}[$\cdot$]\\
	$c_D$				&	Darcy-Weisbach equation coefficient	 				& \hspace{25mm}[$s^2/m$]\\
	$D$					&	Diameter 											& \hspace{25mm}[$m$]\\
	$d$					&	Flow demand	 										& \hspace{25mm}[$m^3/s$]\\
	$f(q)$				&	Pressure drop due to pipe resistance				& \hspace{25mm}[$Pa$]\\
	$h$					&	Pressure drop due to elevation					    & \hspace{25mm}[$Pa$]\\
	$h_{l}$				&	Water level in tanks							    & \hspace{25mm}[$m$]\\
	$h_p$				&	Pressure head									    & \hspace{25mm}[$m$]\\
	$h_t$				&	Total head									        & \hspace{25mm}[$m$]\\
	$J$					&	Mass inertia of water pipes							& \hspace{25mm}[$kgm^2$]\\
	$k_v$				&	Valve conductivity function							& \hspace{25mm}[$\cdot$]\\
	$l$					&	Length (of pipes)									& \hspace{25mm}[$m$]\\
	$p$					&	Absolute pressure									& \hspace{25mm}[$Pa$]\\
	$q$					&	Volumetric flow									    & \hspace{25mm}[$m^3/s$]\\
	$Re$				&	Reynolds number								        & \hspace{25mm}[$\cdot$]\\
	$T$					&	Period of time								        & \hspace{25mm}[$s$]\\
	$z$					&	Elevation head									    & \hspace{25mm}[$m$]\\

	$\gamma$			&	Resistance parameter of pipes 						& \hspace{25mm}[$\cdot$]\\	
	$\Delta p$			&	Differential pressure 						    	& \hspace{25mm}[$Pa$]\\
	$\epsilon$			&	Roughness of pipes 						    		& \hspace{25mm}[$m$]\\	
	$\mu(q,k_v)$		&	Pressure drop on valves 						   	& \hspace{25mm}[$Pa$]\\
	$\omega_r$			&	Impeller rotational speed of centrifugal pumps 		& \hspace{25mm}[$rad/s$]\\
	$\tau$				&	Elevated reservoir parameter 			 			& \hspace{25mm}[$Pa/m^3$]\\	
\end{tabular}

\section*{Constants}

\begin{tabular}{l l l} 
	\textbf{Symbol}		&	\textbf{Description}							& \hspace{64mm}\textbf{Unit}			\\\hline						
	$g$ = 9.83 			&	Gravitational acceleration						& \hspace{64mm}[$m/s^2$]\\
	$\rho = 1000$		&	Density of water								& \hspace{64mm}[$kg/m^3$]\\
	$f_D = 0.05$		&	Darcy friction factor							& \hspace{64mm}[$kg/m^3$]\\
\end{tabular}

\newpage

\section*{Graph theory}

\begin{tabular}{l l} 
	\textbf{Symbol}		&	\textbf{Description}										\\\hline
	$B$		&	Cycle matrix		    												\\
	$\mathcal{E}$		&	Set of edges		    									\\
	$\mathcal{E_T}$		&	Edges regarding the sub-graph $\mathcal{T}$ 		    	\\
	$\mathcal{E_C}$		&	Edges regarding the sub-graph $\mathcal{C}$ 		    	\\
	$I$					&	Identity matrix		    									\\
	$\mathcal{G}$		&	Directed and connected graph								\\
	$H$		&	Incidence matrix		    											\\
	$\mathcal{V}$		&	Set of vertices		    									\\
	$\bar{\mathcal{V}}$ &	Vertices regarding non-inlet points	    					\\
	$\hat{\mathcal{V}}$ &	Vertices regarding inlet points	    						\\
	$c$		&	Number of pumping stations												\\
	$l$		&	Number of elevated reservoirs											\\
	$m$		&	Number of columns in the incidence matrix								\\
	$n$		&	Number of rows in the incidence matrix									\\
	$\mathcal{T}$		&	General sub-graph 		    								\\
	$\mathcal{T^*}$		&	Tree in a graph		    									\\
	$\mathcal{T}^*_{span}$		&	Spanning tree in a graph	    					\\



\end{tabular}	

	

% \section*{Mathematical tools}
% vectorfields
% \\
% time derivatives
% \\
% vectors
% \\
% matrices
% \\
% derivative of vector fields
% \\
% Jakobi matrix
% \\
% chain rule in derivation
% \\
% pseudo inverse
% \\
% explain that $\bm{x}[k]$ is not iteration, it shows that x is a vector and a sequence-> this is very important to state ! 

\newpage
\section*{Glossary of mathematical notation}

\textbf{Matrices and vectors}

Vectors are denoted with lower case letters and matrices are denoted with capital letters such that

\vspace{-6mm}
 \begin{equation*}
 v = 
 \begin{pmatrix}

 		 v_1 	\\
		 v_2 	\\
 		 \vdots \\
		 v_n

 \end{pmatrix}
 \in {\mathbb{R}}^{n},
 \hspace{10mm}
 M = 
 \begin{pmatrix}

 		 m_{11} & m_{12} & \hdots & m_{1k}	\\
 		 m_{21} & m_{22} & \hdots & m_{2k}	\\
 		 \vdots & \vdots & \ddots & \vdots	\\
 		 m_{n1} & m_{n2} & \hdots &m_{nk} \\

 \end{pmatrix}
 \in {\mathbb{R}}^{(n \times k)}.
 \end{equation*}

Diagonal matrices are denoted with $diag(\cdot)$, which maps an n-tuple to the corresponding diagonal matrix:
\vspace{-5mm}

\begin{equation*}
diag \: : \: \mathbb{R}^{n} \rightarrow \mathbb{R}^{(n \times n)},
\end{equation*}
\begin{equation*}
diag(m_{11},\: ... \:, m_{nn} ) := 
\begin{pmatrix}
    m_{11} & & \\
    & \ddots & \\
    & & m_{nn}
  \end{pmatrix}.
  \end{equation*}

A positive semi-definite (resp. positive definite) matrix $A$ is is denoted $A \succeq 0$ ($resp.\; A \; \succ 0$).

The pseudo inverse of a matrix M is denoted with $M^{\dagger}$.

\textbf{Description of non-linear models}

In the thesis we are dealing with dynamical systems that are modelled by a finite number of coupled first-order ordinary differential equations such that 

\vspace{-4mm}
\begin{equation*}
\begin{split}
  \label{non-inlet_p1}
  \dot{x}_1 = & f_1(t,x_1, ..., x_n,u_1,...,u_p) \\
  \dot{x}_2 = & f_2(t,x_1, ..., x_n,u_1,...,u_p) \\
   &\mathrel{\makebox[\widthof{=}]{\vdots}} \\
   \dot{x}_n = & f_n(t,x_1, ..., x_n,u_1,...,u_p)
\end{split}
\end{equation*}

where $\dot{x}_i$ denotes the derivative of $x_i$ with respect to time. The variables in the argument are states $x_i$ and inputs $u_i$. Furthermore, in the thesis, vector notation is used to write the system description in a compact form such that 

\vspace{-4mm}
\begin{equation*}
\label{example1_signals_constants}
x =
 \begin{pmatrix} 
 x_1 \\[1.3pt] 
 x_2 \\[1.3pt]
 \vdots \\[1.3pt] 
 x_n \\[1.3pt] 
 \end{pmatrix}
 , \hspace{5mm}
u =  \begin{pmatrix} 
 u_1 \\[1pt] 
 u_2 \\[1pt]
 \vdots \\[1pt] 
 u_p \\[1pt] 
 \end{pmatrix}
 , \hspace{5mm}
 f(t,x,u) =  \begin{pmatrix} 
 f_1(t,x,u) \\[1pt] 
 f_1(t,x,u) \\[1pt]
 \vdots \\[1pt] 
 f_n(t,x,u) \\[1pt] 
 \end{pmatrix}.
\end{equation*}

\vspace{-2mm}
For the complete model description we use $g(\cdot)$ state model as one n-dimensional first-order vector differential equation. For the output description we use $h(\cdot)$ q-dimensional output model. 

\vspace{-6mm}
\begin{equation*}
\dot{x} = g(t,x,u)
\end{equation*}
\vspace{-8mm}
\begin{equation*}
y = h(t,x,u)
\end{equation*}
\vspace{-6mm}

Typically, for the state equation $f(\cdot)$ is used, however $f(\cdot)$ denotes friction terms in the thesis, therefore is the distinction. 

The vector valued functions are defined such that the mapping is the following

\vspace{-4mm}
 \begin{equation*}
 \alpha(v) : {\mathbb{R}}^{n} \rightarrow {\mathbb{R}}^{n} : [v_1, v_2, \hdots, v_n] \rightarrow [\alpha(v_1), \alpha(v_2),\hdots,\alpha(v_n)].
 \end{equation*}

 \textbf{Indexes}

 Indexes on terms belonging to vectors or matrices are denoted with subscripts. The discrete time index of variables is denoted with $k$ in superscript such that 

\vspace{-6mm}
\begin{equation*}
 x = 
 \begin{pmatrix}

 		 x^k 	\\
		 x^{k+1} 	\\
 		 \vdots \\
		 x^r

 \end{pmatrix},
 \end{equation*}
 \vspace{-6mm}

 where $x^k$ is the $k^{th}$ time step value. 

% \textbf{Upper and lower bounds of a variable}

% \begin{equation}
% \underline{x} < x < \overline{x} 
% \end{equation}

%  Where $x \in {\mathbb{R}} $ and $\overline{x}$ and $\underline{x}$ are the upper and lower bounds.
 
%  \textbf{Intervals}

% \begin{equation}
% 	[a,b] =  \{x \in \mathbb{R}|a\leq x \leq b|\}
% \underline{x} < x < \overline{x} 
% \end{equation}

%  Where $\overline{x}$ and $\underline{x}$ are the upper and lower bounds.

%  \textbf{Vectors and matrices}

% Vectors and matrices are noted with bold fonts, such that $\bm{v}$ is a vector:

% \begin{equation}
% \bm{v} = 
% \begin{bmatrix}

% 		 v_1 	\\
% 		 v_2 	\\
% 		 \vdots \\
% 		 v_n

% \end{bmatrix}
% \in \pmb{{\mathbb{R}}}^{(n \times 1)}
% \end{equation}

% and $\bm{M}$ is a matrix:

% \begin{equation}
% \bm{M} = 
% \begin{bmatrix}

% 		 m_{11} & m_{12} & \hdots & m_{1k}	\\
% 		 m_{21} & m_{22} & \hdots & m_{2k}	\\
% 		 \vdots & \vdots & \ddots & \vdots	\\
% 		 m_{n1} & m_{n2} & \hdots &m_{nk} \\

% \end{bmatrix}
% \in \pmb{{\mathbb{R}}}^{(n \times k)}
% \end{equation}

% Continues vector variables are noted with $\bm{v(t)}$ such that:

% \begin{equation}
% \bm{v(t)} = 
% \begin{bmatrix}

% 		 v_1(t) 	\\
% 		 v_2(t)	\\
% 		 \vdots \\
% 		 v_n(t)

% \end{bmatrix}
% \in \pmb{{\mathbb{R}}}^{(n \times 1)}
% \end{equation}

% While discrete vector variables are called as sequences and are noted with $\bm{v[k]}$, such that:

% \begin{equation}
% \bm{v[k]} = 
% \begin{bmatrix}

% 		 v_1[k] 	\\
% 		 v_2[k]	\\
% 		 \vdots \\
% 		 v_n[k]

% \end{bmatrix}
% \in \pmb{{\mathbb{R}}}^{(n \times 1)}
% \end{equation}

% Where $k$ is the time step between two entries of the sequence.

% The pseudo inverse of a matrix is denoted with $\bm{{M}^{\dagger}}$.

%  \textbf{Small-signal and operating values of signals}

% Small-signals are denoted with $\hat{u}$ and the operating point values are denoted with $\bar{u}$.

%  \textbf{Derivatives}

%  The partial derivative of a function is noted with

% \begin{equation}
% \frac{\partial{f(x,y)}}{\partial{x}}
% \end{equation}

% The derivative of a vector by vector is noted as:

% \begin{equation}
% \frac{\partial{\bm{v}}}{\partial{\bm{w}}} =
% \begin{bmatrix}
%     \frac{\partial v_{1}}{\partial w_1} & \frac{\partial v_{1}}{\partial w_2} &  \dots  & \frac{\partial v_{1}}{\partial w_n} \\
%     \frac{\partial v_{2}}{\partial w_1} & \frac{\partial v_{2}}{\partial w_2} &  \dots  & \frac{\partial v_{2}}{\partial w_n} \\
%     \vdots & \vdots &  \ddots & \vdots \\
%     \frac{\partial v_{k}}{\partial w_1} & \frac{\partial v_{k}}{\partial w_2} &  \dots  & \frac{\partial v_{k}}{\partial w_n}
% \end{bmatrix}
% \end{equation}

% If the size of vector $\bm{v}$ and $\bm{w}$ are the same, the resulting matrix is referred to as a Jacobian matrix in the report.

% The time derivative of a function is noted with

% \begin{equation}
% \dot{f} = \frac{d f(t)}{dt}
% \end{equation}

% \textbf{Vector fields}

% Vector fields are introduced that represent vector valued functions such that the mapping is the following:

% \begin{equation}
% \alpha(\bm{v}) : \pmb{{\mathbb{R}}}^{(n)} \rightarrow \pmb{{\mathbb{R}}}^{(n)} : [v_1, v_2, \hdots, v_n] \rightarrow [\alpha(v_1), \alpha(v_2),\hdots,\alpha(v_n)]
% \end{equation}



