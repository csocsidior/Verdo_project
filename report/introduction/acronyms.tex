\chapter{Nomenclature}

\section*{Acronyms}
	
	\begin{tabular}{|l c l|} \hline
		CT      &&	Circuit Theory									\\ \hline
		D-W     &&	Darcy-Weisbach									\\ \hline
		EPA     &&	Environmental Protection Agency					\\ \hline
		FCV     &&	Flow Control Valve								\\ \hline
		GIS     &&	Geographic Information Systems					\\ \hline
		GT      &&	Graph Theory									\\ \hline
		MPC 	&&	Model Predictive Control						\\ \hline		
		PRV     &&	Pressure Regulating Valve						\\ \hline
		OD		&&	Opening Degree									\\ \hline
		WSS 	&&	Water Supply System								\\ \hline
	\end{tabular}

\section*{Acronyms - Randers Network}
	
	\begin{tabular}{|l c l|} \hline
		BKV     &&	Bunkedal Vandværk 					\\ \hline
		LZ      &&	Low Zone												\\ \hline
		HBP     &&	Hobrovej Pumpestation 		\\ \hline
		HNP     &&	Hornbæk Pumpestation 			\\ \hline
		HSP 	&&	Hadsundvej Pumpestation 	\\ \hline		
		HZ      &&	High Zone												\\ \hline
		OMV		&&	Oust Mølle Vandværk 				\\ \hline
		TBP 	&&	Toldbodgade Pumpestation 	\\ \hline
		VSV 	&&	Vilstrup Vandværk 					\\ \hline
		ØSV 	&&	Østrup Skov Vandværk 			\\ \hline
	\end{tabular}

\newpage

\section*{Symbols}


\begin{tabular}{l l l} 
	\textbf{Symbol}		&	\textbf{Description}								& \hspace{25mm}\textbf{Unit}	\\\hline

	$a_{h2},a_{h1},a_{h0}$ &	Pump constants	 								& \hspace{25mm}[$\cdot$]\\
	$c_D$				&	Darcy-Weisbach equation coefficient	 				& \hspace{25mm}[$\frac{s^2}{m}$]\\
	$D$					&	Diameter (of pipes or tanks)						& \hspace{25mm}[$m$]\\
	$d$					&	Flow demand	 										& \hspace{25mm}[$\frac{m^3}{s}$]\\
	$f(q)$				&	Pressure drop due to pipe resistance				& \hspace{25mm}[$Pa$]\\
	$h$					&	Pressure drop due to elevation					    & \hspace{25mm}[$Pa$]\\
	$h_{l}$				&	Water level in tanks							    & \hspace{25mm}[$m$]\\
	$h_p$				&	Pressure head									    & \hspace{25mm}[$m$]\\
	$h_t$				&	Total head									        & \hspace{25mm}[$m$]\\
	$J$					&	Mass inertia of water pipes							& \hspace{25mm}[$kgm^2$]\\
	$k_v$				&	Valve conductivity function							& \hspace{25mm}[$\cdot$]\\
	$l$					&	Length (of pipes)									& \hspace{25mm}[$m$]\\
	$p$					&	Absolute pressure									& \hspace{25mm}[$Pa$]\\
	$q$					&	Volumetric flow									    & \hspace{25mm}[$\frac{m^3}{s}$]\\
	$Re$				&	Reynolds number								        & \hspace{25mm}[$\cdot$]\\
	$z$					&	Elevation head									    & \hspace{25mm}[$m$]\\

	$\gamma p$			&	Resistance parameter of pipes 						& \hspace{25mm}[$\cdot$]\\	
	$\Delta p$			&	Differential pressure 						    	& \hspace{25mm}[$Pa$]\\
	$\epsilon p$		&	Roughness of pipes 						    		& \hspace{25mm}[$m$]\\	
	$\mu(q,k_v)$		&	Pressure drop on valves 						   	& \hspace{25mm}[$Pa$]\\
	$\omega_r$			&	Impeller rotational speed of centrifugal pumps 		& \hspace{25mm}[$\frac{rad}{s}$]\\
	$\tau$				&	Elevated reservoir parameter 			 			& \hspace{25mm}[$\frac{s^2}{kg}$]\\	
\end{tabular}

\section*{Constants}

\begin{tabular}{l l l} 
	\textbf{Symbol}		&	\textbf{Description}							& \hspace{64mm}\textbf{Unit}			\\\hline						
	$g$ = 9.83 			&	Gravitational acceleration						& \hspace{64mm}[$\frac{m}{s^2}$]\\
	$\rho = 1000$		&	Density of water								& \hspace{64mm}[$\frac{kg}{m^3}$]\\
	$f_D = ?$			&	Darcy friction factor							& \hspace{64mm}[$\frac{kg}{m^3}$]\\
\end{tabular}

\section*{Graph theory}

\begin{tabular}{l l} 
	\textbf{Symbol}		&	\textbf{Description}										\\\hline
	$B$		&	Cycle matrix		    										\\
	$\mathcal{E}$		&	Set of edges		    										\\
	$\mathcal{E_T}$		&	Partitioned set of edges 		    										\\
	$\mathcal{E_C}$		&	Partitioned set of edges 		    										\\
	$\mathcal{G}$		&	Directed and connected graph									    			\\
	$H$		&	Incidence matrix		    										\\
	$\mathcal{V}$		&	Set of vertices		    										\\
	$\bar{\mathcal{V}}$ &	Vertices regarding non-inlet points	    										\\
	$\hat{\mathcal{V}}$ &	Vertices regarding inlet points	    										\\
	$m$		&	Number of columns in the incidence matrix									\\
	$n$		&	Number of rows in the incidence matrix									\\
	$\mathcal{T}$		&	General sub-graph 		    										\\
	$\mathcal{T^*}$		&	Tree in a graph		    										\\
	$\mathcal{T}^*_{span}$		&	Spanning tree in a graph	    										\\


\end{tabular}	

	

% \section*{Mathematical tools}
% vectorfields
% \\
% time derivatives
% \\
% vectors
% \\
% matrices
% \\
% derivative of vector fields
% \\
% Jakobi matrix
% \\
% chain rule in derivation
% \\
% pseudo inverse
% \\
% explain that $\bm{x}[k]$ is not iteration, it shows that x is a vector and a sequence-> this is very important to state ! 

% We should write somthing about notation: 
%\newpage
\section*{Glossary of mathematical notation}

Description of the mathematical notation and terminology used in the report.

% \textbf{Upper and lower bounds of a variable}

% \begin{equation}
% \underline{x} < x < \overline{x} 
% \end{equation}

%  Where $x \in {\mathbb{R}} $ and $\overline{x}$ and $\underline{x}$ are the upper and lower bounds.
 
%  \textbf{Intervals}

% \begin{equation}
% 	[a,b] =  \{x \in \mathbb{R}|a\leq x \leq b|\}
% \underline{x} < x < \overline{x} 
% \end{equation}

%  Where $\overline{x}$ and $\underline{x}$ are the upper and lower bounds.

%  \textbf{Vectors and matrices}

% Vectors and matrices are noted with bold fonts, such that $\bm{v}$ is a vector:

% \begin{equation}
% \bm{v} = 
% \begin{bmatrix}

% 		 v_1 	\\
% 		 v_2 	\\
% 		 \vdots \\
% 		 v_n

% \end{bmatrix}
% \in \pmb{{\mathbb{R}}}^{(n \times 1)}
% \end{equation}

% and $\bm{M}$ is a matrix:

% \begin{equation}
% \bm{M} = 
% \begin{bmatrix}

% 		 m_{11} & m_{12} & \hdots & m_{1k}	\\
% 		 m_{21} & m_{22} & \hdots & m_{2k}	\\
% 		 \vdots & \vdots & \ddots & \vdots	\\
% 		 m_{n1} & m_{n2} & \hdots &m_{nk} \\

% \end{bmatrix}
% \in \pmb{{\mathbb{R}}}^{(n \times k)}
% \end{equation}

% Continues vector variables are noted with $\bm{v(t)}$ such that:

% \begin{equation}
% \bm{v(t)} = 
% \begin{bmatrix}

% 		 v_1(t) 	\\
% 		 v_2(t)	\\
% 		 \vdots \\
% 		 v_n(t)

% \end{bmatrix}
% \in \pmb{{\mathbb{R}}}^{(n \times 1)}
% \end{equation}

% While discrete vector variables are called as sequences and are noted with $\bm{v[k]}$, such that:

% \begin{equation}
% \bm{v[k]} = 
% \begin{bmatrix}

% 		 v_1[k] 	\\
% 		 v_2[k]	\\
% 		 \vdots \\
% 		 v_n[k]

% \end{bmatrix}
% \in \pmb{{\mathbb{R}}}^{(n \times 1)}
% \end{equation}

% Where $k$ is the time step between two entries of the sequence.

% The pseudo inverse of a matrix is denoted with $\bm{{M}^{\dagger}}$.

%  \textbf{Small-signal and operating values of signals}

% Small-signals are denoted with $\hat{u}$ and the operating point values are denoted with $\bar{u}$.

%  \textbf{Derivatives}

%  The partial derivative of a function is noted with

% \begin{equation}
% \frac{\partial{f(x,y)}}{\partial{x}}
% \end{equation}

% The derivative of a vector by vector is noted as:

% \begin{equation}
% \frac{\partial{\bm{v}}}{\partial{\bm{w}}} =
% \begin{bmatrix}
%     \frac{\partial v_{1}}{\partial w_1} & \frac{\partial v_{1}}{\partial w_2} &  \dots  & \frac{\partial v_{1}}{\partial w_n} \\
%     \frac{\partial v_{2}}{\partial w_1} & \frac{\partial v_{2}}{\partial w_2} &  \dots  & \frac{\partial v_{2}}{\partial w_n} \\
%     \vdots & \vdots &  \ddots & \vdots \\
%     \frac{\partial v_{k}}{\partial w_1} & \frac{\partial v_{k}}{\partial w_2} &  \dots  & \frac{\partial v_{k}}{\partial w_n}
% \end{bmatrix}
% \end{equation}

% If the size of vector $\bm{v}$ and $\bm{w}$ are the same, the resulting matrix is referred to as a Jacobian matrix in the report.

% The time derivative of a function is noted with

% \begin{equation}
% \dot{f} = \frac{d f(t)}{dt}
% \end{equation}

% \textbf{Vector fields}

% Vector fields are introduced that represent vector valued functions such that the mapping is the following:

% \begin{equation}
% \alpha(\bm{v}) : \pmb{{\mathbb{R}}}^{(n)} \rightarrow \pmb{{\mathbb{R}}}^{(n)} : [v_1, v_2, \hdots, v_n] \rightarrow [\alpha(v_1), \alpha(v_2),\hdots,\alpha(v_n)]
% \end{equation}



