\chapter{System Modelling}
\label{system_modelling}

\emph{This chapter gives a mathematical description of the component modelling. Thus the different physical and mathematical measures of hydraulic systems are introduced. The similarities to electronic network modelling is shown, by explaining the relevant properties of graph theory. In the end, the EPANET-based modelling approach is introduced which is used for simulation within the project.}

\section{Hydraulic component modelling}
\label{hydraulic_component_modelling}

In this section the mathematical relation between pressure and flow will be given for each of the different components in a WSS system to show their nonlinear behaviour. (The purpose here is not the derivation of the different models, it was done in the previous project.)

\section{Graph-based network modelling}
\label{graph_based_network_modelling}



\section{EPANET modelling}
\label{EPANET modelling}

EPANET is an open source software, created by the United States Environmental Protection Agency (EPA) for simulating hydraulic networks \cite{agency2016epanet}. EPANET allows to track the flow of water in each pipe, the pressure at each node and the height of water in each tank. Furthermore, it uses a node-based model approach which means that the components in the network are either treated as nodes or links. Valves, pumps, reservoirs and tanks are considered as nodes due to their fixed geographical location and geodesic level. Pipes are considered as the links between the nodes in the network. Therefore, nodes are termination points for one or more pipes. The end-user consumption flow demand is considered as an attribute of certain nodes. Such nodes are called demand nodes and they have a certain water withdrawal. Attributes of nodes in the network are the geographical and geodesic coordinates, the flow demand, the total, and the available head. \cite{agency2016epanet}  

In EPANET, there is a function to carry out simulations within an extended period. Time patterns can be created that make demands at the nodes vary in a periodic way over the course of the time period. Nodal demands, reservoirs and pump schedules can all have time patterns associated with them, thus making the hydraulic simulation of the network more realistic. In order to create a schedule plan for changing reservoir levels or schedules for the pumping strategy, it is sufficient to have simulation data only at certain time steps. Therefore it is sufficient to solve the network using a set of hourly time steps (snapshots) over a period of 24 hours, and use the static, steady-state solutions for pump scheduling \cite{agency2016epanet}. The main function of EPANET is this, and therefore is used within this project for extended period analysis. During the analysis, pressure and flow values, along with the demand pattern can be simulated for periodic time steps. 

Since all nodes and links in the network have their unique ID, during the project, the name of certain components will always have a reference to the ID in EPANET, for the better and clearer trackability. 









