\chapter{System Modelling}
\label{system_modelling}

\emph{This chapter gives a mathematical description of the component modelling. Thus the different physical and mathematical measures of hydraulic systems are introduced. The similarities to electronic network modelling are shown by explaining the relevant properties of graph theory. The reduced model for multi-inletsystems is introduced first, then the inclusion of tanks is discussed. In the end, the EPANET-based modelling approach is introduced which is used for simulation purposes within this project.}

\section{Hydraulic component modelling}
\label{hydraulic_component_modelling}

In this section the mathematical relation between pressure and flow is given for each component in a WSS system, in order to show their non-linear behaviour. The purpose here is not to derive the different models, rather to introduce the mathematical formalism which describes them.

\eqref{onecomponent} shows the dual variables which describe all two-terminal components in the network 

\begin{equation}
\label{onecomponent}
 \begin{bmatrix}
    \Delta p \\
    q
\end{bmatrix}
=
 \begin{bmatrix}
    p_{in} - p_{out} \\
    q
\end{bmatrix},
\end{equation}

 \begin{minipage}[t]{0.20\textwidth}
where\\
\hspace*{8mm} $\Delta p$ \\
\hspace*{8mm} $q$ \\
\hspace*{8mm} $p_{in}$, $p_{out}$ 
\end{minipage}
\begin{minipage}[t]{0.68\textwidth}
\vspace*{2mm}
is the differential pressure across the elements,\\
is the flow through the element,\\
are the absolute pressures.
is the flow through the element.
\end{minipage}
\begin{minipage}[t]{0.10\textwidth}
\vspace*{2mm}
\textcolor{White}{te}$\unit{m}$\\
\textcolor{White}{te}$\unit{m}$\\
\textcolor{White}{te}$\unit{\frac{l}{s}}$
\end{minipage}

\subsection{Pipe model}
\label{pipe_component}

Pipes in the network are governed by the dynamic equation

\begin{equation}
\label{complete_pipemodel}
  \Delta p_i = J_i \dot{q_i} + f_i(q_i) - \Delta h_i,
\end{equation}

 \begin{minipage}[t]{0.20\textwidth}
where\\
\hspace*{8mm} $J_i$ \\
\hspace*{8mm} $f_i(q_i)$ \\
\hspace*{8mm} $\Delta h_i$ 
\end{minipage}
\begin{minipage}[t]{0.68\textwidth}
\vspace*{2mm}
is the inertia of the pipes,\\ 
is the pressure drop due to friction,\\
is the pressure drop due to geodesic level difference across the two terminals of pipe elements.
\end{minipage}

The dynamics of the pipes are discarded in the project, as it is shown in other works that the small time constant of the pipe dynamics are not dominant in the system, especially if there are elevated reservoirs included \cite{8thsemester_project,kenneth_houe}. Therefore the pressure across pipes can be written as


\begin{equation}
\label{complete_pipemodel1}
  \Delta p_i = f_i(q_i) - \Delta h_i,
\end{equation}

The pressure drop due to friction across the $i^{th}$ edge is a diagonal map where $f: \mathbb{R}^{m} \rightarrow \mathbb{R}^{m} $ is strictly increasing.\footnote{A map $f: \mathbb{R}^{m} \rightarrow \mathbb{R}^{m} $ is strictly increasing if $\langle x-y, f(x)-f(y) \rangle \geq 0$ for every $x,y \in \: \mathbb{R}^{n}$ such that $x \neq y$ \cite{oneinput_paper}.} As it is shown in \eqref{deltap_friction}, $f_i$ describes a flow dependant pressure drop due to the hydraulic resistance

\begin{equation}
  \label{deltap_friction}
  f_i(q_i) = \rho_i |q_i|q_i,
\end{equation}

 \begin{minipage}[t]{0.20\textwidth}
where\\
\hspace*{8mm} $\rho_i > 0$ 
\end{minipage}
\begin{minipage}[t]{0.68\textwidth}
\vspace*{2mm}
is the parameter of the pipes. 
\end{minipage}

In the following sections it is assumed that each $f_i$ has the structure shown in \eqref{deltap_friction}. 

\subsection{Valve model}
\label{valve_component}

(in progress)

\subsection{Pump model}
\label{pump_component}

(in progress)

\subsection{Elevated reservoir model}
\label{elevatedreservoir_component}

(in progress)

\section{Graph-based network modelling}
\label{graph_based_network_modelling}

Graph-based network modelling has the advantage of making use of tools from circuit theory. Most of these tools are developed based on Graph Theory(GT). These methods can be used to model WSSs as directed graphs, where components of the systems, such as valves, pipes, tanks and pumps correspond to edges and each terminal of the network correspond to nodes, or equivalently, to vertices.

In case of WSSs, in order to track the pressure and flow in the desired part of the network, the equation system of the network has to be solved for the desired edges and vertices. The whole network can be described by writing up the equations for all edges in the network, based on the mathematical modelling of the different components in the system, as shown in \secref{hydraulic_component_modelling}. However, in case of complex systems as water networks for large cities, these systems of equations are hard to handle individually and typically cannot be solved explicitly if there are loops in the system. Therefore the properties of GT are not only useful for setting up relations between flow and pressure, but to make handling of a algebraic constraints easier by exploiting the properties of the matrix representation. Thereby making it convenient for implementing it in computer algorithms for iterative solving methods.  

WSSs can be described by a directed and connected graph, such that \cite{graph_intro}: 

\begin{equation}
  \label{Numberofchords}
  \mathcal{G} = \{\mathcal{V}, \mathcal{E} \} ,
\end{equation}

\begin{minipage}[t]{0.2\textwidth}
where\\
\hspace*{8mm} $\mathcal{G} $ \\
\hspace*{8mm} $\mathcal{V} $ \\
\hspace*{8mm} $\mathcal{E} $
\end{minipage}
\begin{minipage}[t]{0.68\textwidth}
\vspace*{2mm}
is a directed and connected graph,\\
is the set of vertices, where $\mathcal{V} = \{v_1, ..., v_n\}$,\\
is the set of edges, where $\mathcal{E} = \{e_1, ..., e_m\}$. 
\end{minipage}

\subsection{Incidence matrix}
\label{incidence_matrix}

The incidence matrix, $H$, of a connected graph, $\mathcal{G}$, is a matrix where the number of rows and columns correspond to the number of vertices and edges, respectively. Therefore $H\in \: \mathbb{R}^{n \times m}$. In case of hydraulic networks, edges are directed in order to keep track of the direction of the flow in the system. 

\begin{equation}
\label{DiGraph}
 H_{i,j} =
		\left\{
		\begin{array}{ll}
		
		1 			&      \text{if the $j^{th}$ edge is incident out of the $i^{th}$ vertex}.	
\\
	    -1          &      \text{if the $j^{th}$ edge is incident into the $i^{th}$ vertex}.
\\
        0           &      \text{if the $j^{th}$ edge is not connected to the $i^{th}$ vertex}.

		\end{array}
		\right.
\end{equation}	

It is worth mentioning that the reduced incidence matrix can be obtained by removing any arbitrary row from $H$. Therefore $H$ always have $(n-1)$ row rank. This statement can be explained by the mass conservation in the network, which is explained in the following section, \secref{kirchhoffs_law}.

\subsection{Cycle matrix}
\label{cycle_matrix}

Purely tree structure of a WSS is not common when considering water distribution systems. However, trees can be arbitrarily chosen from the underlying graph of the system.\footnote{Recall that a tree with $n$ vertices has $n-1$ edges \cite{deo2017graph}.}  A tree, $\mathcal{T} $, of the graph is a connected sub-graph where any two vertices are connected by exactly one path \cite{deo2017graph}. Therefore a certain sub-graph which is a tree of the network can be represented as follows

\begin{equation}
  \label{Numberofchords}
  \mathcal{T} = \{\mathcal{V_{\mathcal{T}}}, \mathcal{E_{\mathcal{T}}} \} 
\end{equation}

A special case of connected tree sub-graphs is the spanning tree of the network. A spanning tree contains all the vertices of $\mathcal{G}$ and has no cycles, since it is a tree. A spanning tree of the network therefore can be represented as

\begin{equation}
  \label{Numberofchords}
  \mathcal{T} = \{\mathcal{V}, \mathcal{E_{\mathcal{T}}} \} 
\end{equation}

In order to obtain a spanning tree, an edge has to be removed from each cycle. The removed edges are $\mathcal{G} - \mathcal{T}$, and called the chords of $\mathcal{T}$ with respect to $\mathcal{G}$. By adding a chord to $\mathcal{T}$, a cycle is created which is called a fundamental cycle. A graph is conformed by as many fundamental cycles as the number of chords \cite{deo2017graph}.

The set of fundamental cycles correspond to the fundamental cycle matrix, $B$, such that the number of rows and columns are defined by the number of chords and edges, respectively. The cycle matrix of the system is given by

\begin{equation}
\label{DiGraphCycle}
 B_{i,j} =
		\left\{
		\begin{array}{ll}
		
		1 			&     \text{if the $j^{th}$ edge belongs to the $i^{th}$ cycle and their directions agree}	
\\
		-1          &     \text{if the $j^{th}$ edge belongs to the $i^{th}$ cycle and their directions are opposite}
\\
        0           &     \text{if the $j^{th}$ edge does not belong to the $i^{th}$ cycle}
		\end{array}
		\right.
\end{equation}	

\subsection{Kirchhoff's and Ohm's law for hydraulic networks}
\label{kirchhoffs_law}

In this project the hydraulic system is considered to be an open network with pipes, valves, pumps and the storage tanks, where water is able to enter and leave the network at a subset of the vertices. For such a system Kirchhoff's vertex law corresponds to conservation of mass in each vertex and described by

\begin{equation}
  \label{vertexlaw_open}
  Hq = d,
\end{equation}

  \begin{minipage}[t]{0.20\textwidth}
where\\
\hspace*{8mm} $d \in \: \mathbb{R}^{n}$ 
\end{minipage}
\begin{minipage}[t]{0.68\textwidth}
\vspace*{2mm}
is the vector of nodal demands, with $d_i > 0$ when demand flow is into vertex $i$ and $d_i < 0$ when demand flow is out of vertex $i$.
\end{minipage}
\begin{minipage}[t]{0.10\textwidth}
\vspace*{2mm}
\textcolor{White}{te}$\unit{\frac{L}{s}}$
\end{minipage}

Nodal demands can be seen as the end-user consumption, which means that water is taken out from the network. The mass conservation corresponds to the fact that what is consumed from the system must also be produced. Due to mass conservation, there can be only $(n-1)$ independent nodal demands in the network

\begin{equation}
  \label{mass_conservation}
  d_n = - \sum_{i=1}^{n-1} d_i.
\end{equation}

In the further project, a distinction is made between inlet and non-inlet nodes. It is assumed that the demand at non-inlet nodes fulfil the following constraint

\begin{equation}
  \label{non_inlet_constraint}
  d_i \geq 0.
\end{equation}

It is worth noting however, that in closed hydraulic networks the vertex law is

\begin{equation}
  \label{vertexlaw_closed}
  Hq = 0.
\end{equation}

Ohm's law for hydraulic networks therefore can be expressed with the incidence matrix, when $H^T$ is applied to the vector of absolute pressures, $p$. Important to point out that the edges of the underlying graph are considered as only pipe elements

\begin{equation}
  \label{ohmslaw}
  \Delta p = H^Tp = f(q) - H^Th.
\end{equation}

In \eqref{ohmslaw} the differential pressure is described across each edge in the network, taking into account the pressure loss due to friction, $f(q)$ and the pressure drop due to geodesic level differences,  where $h \in \: \mathbb{R}^{n}$ is the vector of geodesic levels at each vertex expressed in units of potential, i.e. pressure. 






\subsection{Multi-inlet reduced network model}
\label{multi_inlet_reduced_network_description}

The system is considered to be a water network supplied from more than one pumping stations and several end-users. In the underlying graph therefore the nodes are pipe connections, with possible water demand from the end-users, and the edges are pipes. The inclusion of storage tanks is the next step of the model development, therefore it is described in a following section, in \secref{inclusion_of_reservoirs}.

The aim of the modelling is to obtain a reduced order network model which is able to capture the dependence of the measured output pressures on the flows and pressures at the inlets. Therefore it is assumed that the inlet pressures and demands are measured. Furthermore, pressure measurement is available in the remaining network, at the end-users. Considering generality, the model is described for $c$ inlets, however it should be noted that regarding the Randers WSS, two inlet points are considered. 

In order to put the system into a form which can handle the measured pressure dependencies on the control inputs, the underlying graph of the network is first partitioned. The $n$ vertices of the graph are separated into two sets

\begin{equation}
  \label{vertices1}
  \mathcal{V} = \{\bar{\mathcal{V}}, \hat{\mathcal{V}} \}, 
\end{equation}

\begin{minipage}[t]{0.3\textwidth}
where\\
\hspace*{8mm} $\hat{\mathcal{V}} = \{\hat{v}_1, ..., \hat{v}_c\}$\\
and \\
\hspace*{8mm} $\bar{\mathcal{V}} = \{\bar{v}_1, ..., \bar{v}_{n-c}\}$ 
\end{minipage}
\begin{minipage}[t]{0.55\textwidth}
\vspace*{2mm}
 represents the vertices corresponding to the inlet points,\\
 represents the remaining vertices in the graph.
\end{minipage}

The partitioning for the $m$ edges of the graph is being chosen such that

\begin{equation}
  \label{edges1}
  \mathcal{E} = \{\mathcal{E_{\mathcal{T}}}, \mathcal{E_{\mathcal{C}}} \},
\end{equation}

\begin{minipage}[t]{0.35\textwidth}
where\\
\hspace*{8mm} $\mathcal{E_{\mathcal{T}}} = \{e_{\mathcal{T},1}, ..., e_{\mathcal{T},n-c}\}$\\
and\\
\hspace*{8mm} $\mathcal{E_{\mathcal{C}}} = \{e_{\mathcal{C},1}, ..., e_{\mathcal{C},m-n+c}\}$. 
\end{minipage}

The subsets regarding edges and the partitioning is chosen such that the sub-matrix, which maps edges in $\mathcal{E_{\mathcal{T}}}$ to vertices in $\bar{\mathcal{V}}$, is invertible. 

Therefore the incidence matrix can be split into four sub-matrices, as shown in \eqref{H_matrix_sub} below

\begin{equation}
\label{H_matrix_sub}
H=
\left[
\begin{array}{c;{2pt/2pt}r}
\bar{H}_{\mathcal{T}} & \bar{H}_{\mathcal{C}} \\
\hdashline[2pt/2pt]
\hat{H}_{\mathcal{T}} & \hat{H}_{\mathcal{C}}
\end{array}
\right],
\end{equation}

\begin{minipage}[t]{0.3\textwidth}
where\\
\hspace*{8mm} $\bar{H}_{\mathcal{T}} \in \mathbb{R}^{(n-c) \times (n-c)}$\\ 
\hspace*{8mm} $\bar{H}_{\mathcal{C}} \in \mathbb{R}^{(n-c) \times (m-n+c)}$\\
\hspace*{8mm} $\hat{H}_{\mathcal{T}} \in \mathbb{R}^{c \times (n-c)}$\\
\hspace*{8mm} $\hat{H}_{\mathcal{C}} \in \mathbb{R}^{c \times (m-n+c)}$
\end{minipage}
\begin{minipage}[t]{0.68\textwidth}
\vspace*{-0.1mm}
is the sub-matrix, mapping edges in $\mathcal{E_{\mathcal{T}}}$ to vertices in $\bar{\mathcal{V}}$,\\ 
is the sub-matrix, mapping edges in $\mathcal{E_{\mathcal{C}}}$ to vertices in $\bar{\mathcal{V}}$,\\
is the sub-matrix, mapping edges in $\mathcal{E_{\mathcal{T}}}$ to vertices in $\hat{\mathcal{V}}$,\\
is the sub-matrix, mapping edges in $\mathcal{E_{\mathcal{C}}}$ to vertices in $\hat{\mathcal{V}}$. 
\end{minipage}

It is worth noting that the only requirement for the edge partitioning is $\bar{H}_{\mathcal{T}}$ being invertible\footnote{$\exists \{\mathcal{V}, \mathcal{E} \} : \bar{H}^{-1}_{\mathcal{T}} \because rank(H) = (n-1) $ \cite{deo2017graph} }. Furthermore, the set $\mathcal{T} = \{\mathcal{V}, \mathcal{E_{\mathcal{T}}} \}$ is not necessarily a tree of the underlying graph, it can be any form of a connected graph that fulfils the requirements. However, one special case is when $c = 1$, meaning that the network has only one inlet. In this case, $\mathcal{T}$ is indeed a spanning tree. 

With the chosen partition, Kirchhoff's vertex law in \eqref{vertexlaw_open} can be rewritten as

\begin{equation}
  \label{vertexlaw_partitioned1}
  \bar{d} = \bar{H}_{\mathcal{T}} q_{\mathcal{T}} + \bar{H}_{\mathcal{C}} q_{\mathcal{C}},
\end{equation}

\begin{equation}
  \label{vertexlaw_partitioned2}
  \hat{d} = \hat{H}_{\mathcal{T}} q_{\mathcal{T}} + \hat{H}_{\mathcal{C}} q_{\mathcal{C}},
\end{equation}

and Ohm's law in \eqref{ohmslaw}, separating the pressure drop due to hydraulic resistance

\begin{equation}
  \label{ohmslaw_partitioned1}
  f_{\mathcal{T}}(q_\mathcal{T}) = \bar{H}^T_{\mathcal{T}} (\bar{p} + \bar{h}) + \hat{H}^T_{\mathcal{T}} (\hat{p} + \hat{h}),
\end{equation}

\begin{equation}
  \label{ohmslaw_partitioned2}
  f_{\mathcal{C}}(q_\mathcal{C}) = \bar{H}^T_{\mathcal{C}} (\bar{p} + \bar{h}) + \hat{H}^T_{\mathcal{C}} (\hat{p} + \hat{h}).
\end{equation}



\subsection{Inclusion of elevated reservoirs}
\label{inclusion_of_reservoirs}

As it is described in \eqref{non_inlet_constraint}, a distinction is made between non-inlet and inlet vertices, by assuming that non-inlet vertices have only positive or zero nodal demand. However, when the inclusion of a tank is considered, a special type of node has to be introduced. A node which can have a demand in both positive and negative directions, meaning that the demand is positive when the tank is being filled and negative when it is being emptied.



\section{EPANET modelling}
\label{EPANET_modelling}

EPANET is an open source software, created by the United States Environmental Protection Agency (EPA) for simulating hydraulic networks \cite{agency2016epanet}. EPANET allows to track the flow of water in each pipe, the pressure at each node and the height of water in each tank. Furthermore, it uses a node-based model approach which means that the components in the network are either treated as nodes or links. Valves, pumps, reservoirs and tanks are considered as nodes due to their fixed geographical location and geodesic level. Pipes are considered as the links between the nodes in the network. Therefore, nodes are termination points for one or more pipes. The end-user consumption flow demand is considered as an attribute of certain nodes. Such nodes are called demand nodes and they have a certain water withdrawal. Attributes of nodes in the network are the geographical and geodesic coordinates, the flow demand, the total, and the available head. \cite{agency2016epanet}  

In EPANET, there is a function to carry out simulations within an extended period. Time patterns can be created that make demands at the nodes vary in a periodic way over the course of the time period. Nodal demands, reservoirs and pump schedules can all have time patterns associated with them, thus making the hydraulic simulation of the network more realistic. In order to create a schedule plan for changing reservoir levels or schedules for the pumping strategy, it is sufficient to have simulation data only at certain time steps. Therefore it is sufficient to solve the network using a set of hourly time steps (snapshots) over a period of 24 hours, and use the static, steady-state solutions for pump scheduling \cite{agency2016epanet}. The main function of EPANET is this, and therefore is used within this project for extended period analysis. During the analysis, pressure and flow values, along with the demand pattern can be simulated for periodic time steps. 

Since all nodes and links in the network have their unique IDs, during the project, the name of certain components will always have a reference to the original IDs in EPANET, for the better and clearer trackability. 









