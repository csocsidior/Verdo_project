\chapter{System Modelling}
\label{system_modelling}

\emph{This chapter gives a mathematical description of the component modelling. Thus the different physical and mathematical measures of hydraulic systems are introduced. The similarities to electronic network modelling are shown by explaining the relevant properties of graph theory. In the end, the EPANET-based modelling approach is introduced which is used for simulation purposes within this project.}

\section{Hydraulic component modelling}
\label{hydraulic_component_modelling}

In this section the mathematical relation between pressure and flow will be given for each of the different components in a WSS system to show their nonlinear behaviour. (The purpose here is not the derivation of the different models, it was done in the previous project.)

\section{Graph-based network modelling}
\label{graph_based_network_modelling}

Graph-based network modelling has the advantage of making use of tools from circuit theory. Most of these tools are developed based on Graph Theory(GT). These methods can be used to model WSSs as directed graphs, where components of the systems, such as valves, pipes, tanks and pumps correspond to edges and each terminal of the network correspond to nodes, or equivalently, to vertices.

In case of WSSs, in order to track the pressure and flow in the desired part of the network, the equation system of the network has to be solved for the desired edges or vertices. The whole network can be described by writing up the equations for all vertices in the network, based on the mathematical modelling of the different components in the system, as shown in \secref{hydraulic_component_modelling}. However, in case of complex systems as a water network for a large cities, these systems of equations are hard to handle individually and typically cannot be solved explicitly if there are loops in the system. Therefore the properties of GT are not only useful for setting up relations between flow and pressure, but to make handling of a complex system easier by exploiting the properties of matrix representations. Thereby making it convenient for implementing it in computer algorithms for iterative solving methods.  

WSSs can be described by a directed and connected graph, such that \cite{graph_intro}: 

\begin{equation}
  \label{Numberofchords}
  \mathcal{G} = \{\mathcal{V}, \mathcal{E} \} 
\end{equation}

\begin{minipage}[t]{0.20\textwidth}
Where\\
\hspace*{8mm} $\mathcal{G} $ \\
\hspace*{8mm} $\mathcal{V} $ \\
\hspace*{8mm} $\mathcal{E} $
\end{minipage}
\begin{minipage}[t]{0.68\textwidth}
\vspace*{2mm}
\hspace*{8mm} is a directed and connected graph,\\
\hspace*{8mm} is the set of vertices, where $\mathcal{V} = \{v_1, ..., v_n\}$,\\
\hspace*{8mm} is the set of edges, where $\mathcal{E} = \{e_1, ..., e_m\}$. 
\end{minipage}

\subsection{Incidence matrix}
\label{incidence_matrix}

The incidence matrix, $H$, of a connected graph, $\mathcal{G}$, is a matrix where the number of rows and columns correspond to the number of vertices and edges, respectively. Therefore $H\in \: \mathbb{R}^{n \times m}$. In case of hydraulic networks, edges are directed in order to keep track of the direction of the flow in the system. 

\begin{equation}
\label{DiGraph}
 H_{i,j} =
		\left\{
		\begin{array}{ll}
		
		1 			&      \text{if the $j^{th}$ edge is incident out of the $i^{th}$ vertex}.	
\\
	    -1          &      \text{if the $j^{th}$ edge is incident into the $i^{th}$ vertex}.
\\
        0           &      \text{if the $j^{th}$ edge is not connected to the $i^{th}$ vertex}.

		\end{array}
		\right.
\end{equation}	

\subsection{Cycle matrix}
\label{cycle_matrix}

Purely tree structure of a WSS is not common when considering distribution systems for whole cities. However, trees can be arbitrarily chosen from the underlying graph of the system. A connected tree, $\mathcal{T} $, of the graph is a connected sub-graph where any two vertices are connected by exactly one path \cite{deo2017graph}. Therefore a certain sub-graph which is a tree of the network can be represented as follows

\begin{equation}
  \label{Numberofchords}
  \mathcal{T} = \{\mathcal{V_{\mathcal{T}}}, \mathcal{E_{\mathcal{T}}} \} 
\end{equation}

A special case of connected tree sub-graphs is the spanning tree of the network. A spanning tree contains all the vertices of $\mathcal{G}$ and has no cycles, since it is a tree. A spanning tree of the network therefore can be represented as

\begin{equation}
  \label{Numberofchords}
  \mathcal{T} = \{\mathcal{V}, \mathcal{E_{\mathcal{T}}} \} 
\end{equation}

In order to obtain a spanning tree, an edge has to be removed from each cycle. The removed edges are $\mathcal{G} - \mathcal{T}$, and called the chords of $\mathcal{T}$ with respect to $\mathcal{G}$. By adding a chord to $\mathcal{T}$, a cycle is created which is called a fundamental cycle. A graph is conformed by as many fundamental cycles as the number of chords \cite{deo2017graph}.

The set of fundamental cycles correspond to the fundamental cycle matrix, $B$, such that the number of rows and columns are defined by the number of chords and edges, respectively. The cycle matrix of the system is given by

\begin{equation}
\label{DiGraphCycle}
 B_{i,j} =
		\left\{
		\begin{array}{ll}
		
		1 			&     \text{if the $j^{th}$ edge belongs to the $i^{th}$ cycle and their directions agree}	
\\
		-1          &     \text{if the $j^{th}$ edge belongs to the $i^{th}$ cycle and their directions are opposite}
\\
        0           &     \text{if the $j^{th}$ edge does not belong to the $i^{th}$ cycle}
		\end{array}
		\right.
\end{equation}	

\subsection{Kirchhoff's and Ohm's law}
\label{kirchhoffs_law}

(in progress)

\subsection{Partitioning the network}
\label{partitioning_the_network}

(in progress)
Show it on a small example.

\section{EPANET modelling}
\label{EPANET_modelling}

EPANET is an open source software, created by the United States Environmental Protection Agency (EPA) for simulating hydraulic networks \cite{agency2016epanet}. EPANET allows to track the flow of water in each pipe, the pressure at each node and the height of water in each tank. Furthermore, it uses a node-based model approach which means that the components in the network are either treated as nodes or links. Valves, pumps, reservoirs and tanks are considered as nodes due to their fixed geographical location and geodesic level. Pipes are considered as the links between the nodes in the network. Therefore, nodes are termination points for one or more pipes. The end-user consumption flow demand is considered as an attribute of certain nodes. Such nodes are called demand nodes and they have a certain water withdrawal. Attributes of nodes in the network are the geographical and geodesic coordinates, the flow demand, the total, and the available head. \cite{agency2016epanet}  

In EPANET, there is a function to carry out simulations within an extended period. Time patterns can be created that make demands at the nodes vary in a periodic way over the course of the time period. Nodal demands, reservoirs and pump schedules can all have time patterns associated with them, thus making the hydraulic simulation of the network more realistic. In order to create a schedule plan for changing reservoir levels or schedules for the pumping strategy, it is sufficient to have simulation data only at certain time steps. Therefore it is sufficient to solve the network using a set of hourly time steps (snapshots) over a period of 24 hours, and use the static, steady-state solutions for pump scheduling \cite{agency2016epanet}. The main function of EPANET is this, and therefore is used within this project for extended period analysis. During the analysis, pressure and flow values, along with the demand pattern can be simulated for periodic time steps. 

Since all nodes and links in the network have their unique IDs, during the project, the name of certain components will always have a reference to the original IDs in EPANET, for the better and clearer trackability. 









