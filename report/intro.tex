\chapter{Introduction}
\label{introduction}

Due to the fast-paced technological development all over the world, the demand for industrial growth and energy resources has seen a rapid increase. Along with the industrial growth, the sudden rise in population has made the world realize that shortage of energy sources is an actual and universally anticipated problem \cite{sustainable_water}. In order to cope with such issues and to make the rapid development possible and less expensive, the world is moving towards more efficient use of resources and optimization of infrastructure. Therefore, technological development is also moving the focus on green energy, resulting in more and more renewable energy sources added to the grid \cite{fluctuating_price}. 

Water Supply Systems(WSSs) are among the leading sectors of the industrial growth. Besides that, WSSs are one of the most vital infrastructures of modern societies in the world. Typically, such networks are operating by making pumps transport water from reservoirs through the pipe network, to the end-users. In most cases, elevated reservoirs are exploited in these WSSs, such that they can even out the demand differences for the consumers. Although elevated reservoirs are usually an integrated part of these systems, providing drinking water is a highly energy-intensive activity. For instance, in the United States alone, the drinking water and waste water systems are typically the largest energy consumers, accounting for 25 to 40 percent of a municipality's total energy bill \cite{appelbaum2002water}. 

Since fresh water is limited, and due to the presence of global changes such as climate change and urbanization, new trends are emerging in the water supply sector. In the past few decades, several research and case study showed that WSSs and other energy distribution networks need to be improved due to the leakages in the system, high cost of maintenance and due to high energy consumption. Companies also realized that by using proper pressure management in their networks, the effect of leakages can be reduced, thereby huge amount of fresh water can be saved \cite{national2005public}. 

In Denmark recently, the larger water suppliers have been focusing on making the water supply sector more effective through introducing a benchmarking system focusing on the environment, the security of supply and the efficiency based on user demands. Since 1980, these efficiency activities has been an important issue \cite{water_denmark}. It has been proved that by utilizing advanced, energy- or cost-optimizing control schemes and utilizing renewable energy sources, such as elevated reservoirs, the life of the existing infrastructure can be extended and money or energy can be saved \cite{sustainable_water}. Therefore there is a growing demand in industry for developing methods, leading towards more efficient WSSs. 

The presented project is executed in collaboration with the company, Verdo A/S. It is in the interest of Verdo A/S to utilize an advanced model-based optimal control scheme on the WSS with several storages in Randers, Denmark. For a large municipality such as Randers, the water distribution network is complex and consists of thousands of elements. Since the control algorithm itself is complex and model-based, the computational effort is also high. Furthermore, the offline optimisation of a large-scale WSS means that any changes to the network may require significant changes in the optimisation method, which leads to high costs of the system maintenance \cite{brdys1994operational}. Therefore typically a reduction is required in such networks to make the online execution of the control algorithm possible. 

The long-term goal of this project is to find a solution for implementing Model Predictive Control(MPC) on the Randers WSS. However, before the implementation of any control scheme would be possible, a proper and identified model is required. Therefore, as the first part of the project, the following problem statement can be formulated: 

\emph{How can the WSS in Randers be simplified and identified, with storages included in the system, such that the reduced model preserves the original nonlinear behaviour and remains suitable for a plug-and-play commissionable Model Predictive Control scheme.}









