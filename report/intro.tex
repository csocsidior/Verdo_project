\chapter{Introduction}
\label{introduction}

Due to the fast-paced technological development all over the world, the demand for industrial growth and for energy resources has seen a rapid increase. Along with the industrial growth, the sudden rise in population has made the world realize that shortage of energy sources is an actual and universally anticipated problem \cite{sustainable_water}. In order to cope with such shortage issues and to make the rapid development possible and less expensive, the world is moving towards more efficient use of resources and optimization of the infrastructure. Therefore, technological development is also moving the focus to green energy and to solutions for more optimized operation of the existing infrastructure, resulting in more and more renewable energy sources added to the grid and smarter solutions for control \cite{fluctuating_price}. 

Water Supply Systems(WSSs) are among the sectors which make the industrial growth possible. On top of this, WSSs are one of the most vital infrastructures of modern societies in the world. In Denmark typically, such networks are operating by making pumps transport water from reservoirs through the pipe network, to the consumers. In most cases, elevated reservoirs are utilized in these WSSs, such that they can even out the demand differences for the consumers. Although elevated reservoirs are usually an integrated part of these systems, providing drinking water is a highly energy-intensive activity. For instance, in the United States alone, the drinking water and waste water systems are typically the largest energy consumers, accounting for 25 to 40 percent of a municipality's total public expenditure. \cite{appelbaum2002water}. 

Since fresh water is limited, and due to the presence of global changes such as climate change and urbanization, new trends are emerging in the water supply sector. In the past few decades, several research and case study showed that the operation of WSSs and other energy distribution networks need to be improved due to the leakages in the system, the high cost of maintenance and due to high energy consumption. Companies which are operating WSSs also realized that by using proper pressure management in their networks, the effect of leakages can be reduced, thereby huge amount of fresh water and money can be saved annually\cite{national2005public}. 

In Denmark recently, the larger water suppliers have been focusing on making the water supply sector more effective through introducing a benchmarking system. This system is focusing on the environment, the security of supply and the efficiency based on user demands. Since 1980, these efficiency activities has been an important issue \cite{water_denmark}. It has been proved that by utilizing advanced, energy- or cost-optimizing control solutions and utilizing renewable energy sources such as elevated reservoirs, the lifetime of the existing infrastructure can be extended and money or energy can be saved \cite{sustainable_water}. Therefore, there is a growing demand in industry for developing methods, leading towards more efficient WSSs. 

The presented project is executed in collaboration with the company, Verdo A/S. It is in the interest of Verdo A/S to utilize an advanced model-based optimal control scheme on the WSS with several storages in their system in Randers, Denmark. For a large municipality such as Randers, the water distribution network is complex and consists of thousands of hydraulic elements. Since an online optimizing control algorithm is considered to be complex, the computational effort of such algorithms is also high. Furthermore, offline optimisation of large-scale WSSs means that any modifications in the existing network may require significant changes in the optimisation method, which leads to high costs of system maintenance \cite{brdys1994operational}. In fact, modifications in a water distribution network are not unusual, as with time passing by, more and more residential areas are being covered, requiring the extension of the existing pipelines. Therefore, a well-describing, reduced model is required, on which the capability of executing complex, online control algorithms is a good possibility. 

The long-term goal of the thesis is to find a solution for implementing Model Predictive Control(MPC) on the Randers WSS. However, before the implementation of any control scheme would be possible, a reduced and well-describing model is required. The thesis uses the results presented in \cite{oneinput_paper} and extends it such that a mathematical model is able to describe the dynamics of multiple elevated reservoirs or equivalently Water Tanks(WTs) in the network.  Therefore, the main goal of the thesis can be summarized in the following problem statement:

\emph{How can the WSS in Randers be simplified and identified, taking into account the presence of water storages, such that a reduced model of the system preserves the original non-linear behaviour and remains suitable for a plug-and-play commissionable Model Predictive Control scheme.}









