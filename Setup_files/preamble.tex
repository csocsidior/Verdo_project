\documentclass[a4paper,11pt,fleqn,dvipsnames,twoside,openright]{memoir}  	% Openright aabner kapitler paa hoejresider (openany begge)


%%%% PACKAGES %%%%
% ¤¤ Oversaettelse og tegnsaetning ¤¤ %
\usepackage[utf8]{inputenc}					% Input-indkodning af tegnsaet (UTF8)
\usepackage[english]{babel}					% Dokumentets sprog
\usepackage[OT1]{fontenc}					% Output-indkodning af tegnsaet (T1)
\usepackage{ragged2e,anyfontsize}			% Justering af elementer
\usepackage{mfirstuc}

% ¤¤ Packages for the solar panel figure ¤¤ %
\usepackage{xcolor}
\usepackage[american,cuteinductors,smartlabels]{circuitikz}
\usepackage{siunitx}
\usetikzlibrary{calc}
\usetikzlibrary{patterns}
\ctikzset{bipoles/thickness=1}
\ctikzset{bipoles/length=0.8cm}
\ctikzset{bipoles/diode/height=.375}
\ctikzset{bipoles/diode/width=.3}
\ctikzset{tripoles/thyristor/height=.8}
\ctikzset{tripoles/thyristor/width=1}
\ctikzset{bipoles/vsourceam/height/.initial=.7}
\ctikzset{bipoles/vsourceam/width/.initial=.7}
\tikzstyle{every node}=[font=\small]
\tikzstyle{every path}=[line width=0.8pt,line cap=round,line join=round]

\definecolor{AleeRed}{rgb}{0.5,0,0}
\definecolor{gray75}{gray}{0.75}
\usepackage{graphicx}
\usepackage{adjustbox}
\usepackage{epstopdf}

%Roman letters
\ExplSyntaxOn
\NewDocumentCommand{\RN}{m}
 {
  \textup{ \int_to_Roman:n { #1 } }
 }
\ExplSyntaxOff

																			
% ¤¤ Figurer og tabeller (floats) ¤¤ %
\usepackage{graphicx} 						% Haandtering af eksterne billeder (JPG, PNG, EPS, PDF)
\usepackage{multirow}                		% Fletning af raekker og kolonner (\multicolumn og \multirow)
\usepackage{multicol}         	        	% Muliggoer output i spalter
\usepackage{rotating}						% Rotation af tekst med \begin{sideways}...\end{sideways}
\usepackage{colortbl} 						% Farver i tabeller (fx \columncolor og \rowcolor)

\usepackage{flafter}						% Soerger for at floats ikke optraeder i teksten foer deres reference
\let\newfloat\relax 						% Justering mellem float-pakken og memoir
\usepackage{float}							% Muliggoer eksakt placering af floats, f.eks. \begin{figure}[H]
\newcommand{\HRule}{\rule{\linewidth}{0.5mm}}
\usepackage{pdflscape}	 					% Giver mulighed for landacape mode
\usepackage{longtable}						% Giver mulighed for talber over flere side
%\usepackage{subcaption}
\usepackage{capt-of}

% ¤¤ Matematik mm. ¤¤
\usepackage{amsmath}						% Avancerede matematik-udvidelser
\usepackage{blkarray}
\usepackage{amssymb}						% Avancerede matematik-udvidelser
\usepackage{stmaryrd}						% Avancerede matematik-udvidelser
\usepackage{amsthm}							% Avancerede matematik-udvidelser
\usepackage{textcomp}                 		% Symbol-udvidelser (f.eks. promille-tegn med \textperthousand )
\usepackage{mathtools}
\usepackage{lmodern,bm}                
\usepackage[T1]{sansmath} 
\SetMathAlphabet{\mathsfbf}{sans}{\sansmathencoding}{\sfdefault}{bx}{sl}
\usepackage{etoolbox}
\AtBeginEnvironment{sansmath}{\let\bm\mathsfbf}{}{}
\DeclareRobustCommand{\ttfamily}{\fontencoding{T1}\fontfamily{lmtt}\selectfont}


\usepackage{gensymb}

% ¤¤ Kode indsat i rapport ¤¤ %
\usepackage{listings}						% Placer kildekode i dokumentet med \begin{lstlisting}...\end{lstlisting}



\usetikzlibrary{decorations.pathmorphing}
%\usepackage[detect-all]{siunitx}

\tikzset{
   ragged border/.style={ decoration={random steps, segment length=1mm, amplitude=0.5mm},
           decorate,
   }
}

  \usetikzlibrary{arrows}
\usepackage{array}
\newcolumntype{P}[1]{>{\centering\arraybackslash}p{#1}}
\newcolumntype{M}[1]{>{\centering\arraybackslash}m{#1}}


% ¤¤ Matematik funktioner ¤¤				% Sikre alle funktioner står med ordinær skriftype
	\newcommand{\e}{\text{e}}
	\newcommand{\dB}{\text{dB}}	
	\newcommand{\I}{\text{I}}	
	\newcommand{\V}{\text{V}}
	\newcommand{\A}{\text{A}}
	\newcommand{\W}{\text{W}}
	\newcommand{\F}{\text{F}}
	\renewcommand{\S}{\text{S}}	
	\newcommand{\rms}{\text{rms}}
	\newcommand{\m}{\text{m}}
	\newcommand{\n}{\text{n}}
	\newcommand{\p}{\text{p}}
	\newcommand{\f}{\text{f}}
	\renewcommand{\k}{\text{k}}
	\newcommand{\M}{\text{M}}
	\newcommand{\G}{\text{G}}
	\newcommand{\C}{\text{C}}
	\renewcommand{\c}{\text{c}}
	\newcommand{\T}{\text{T}}
	\newcommand{\km}{\text{km}}
	\newcommand{\h}{\text{h}}		
	\newcommand{\s}{\text{s}}
	\newcommand{\cm}{\text{cm}}
	\newcommand{\ms}{\text{ms}}


% ¤¤ Opsætning af enheder ¤¤ % 
	\newcommand{\unit}[1]{\hfill\left[\mathrm{ #1}\right]} 	% Skriv \enhed{"din enhed"} og dem bliver rykket til højre


% ¤¤ Referencer og kilder ¤¤ %
			
\usepackage[square, numbers, comma, sort&compress]{natbib}			
\bibliographystyle{ieeetr}
% Udseende af litteraturlisten.

% ¤¤ Textlable setup ¤¤ % 
\makeatletter
\newcommand*{\textlabel}[2]{				% Gør \textlabel mulig
  \edef\@currentlabel{#1}					% Set target label
  \phantomsection							% Correct hyper reference link
  #1\label{#2}}								% Print and store label

% ¤¤ Misc. ¤¤ %
\usepackage{lipsum}							% Dummy text \lipsum[..]
\usepackage[shortlabels]{enumitem}			% Muliggoer enkelt konfiguration af lister
\usepackage{pdfpages}						% Goer det muligt at inkludere pdf-dokumenter med kommandoen \includepdf[pages={x-y}]{fil.pdf}	
\pdfoptionpdfminorversion=6					% Muliggoer inkludering af pdf dokumenter, af version 1.6 og hoejere
\pretolerance=2500 							% Justering af afstand mellem ord (hoejt tal, mindre orddeling og mere luft mellem ord)

% Kommentarer og rettelser med \fxnote. Med 'final' i stedet for 'draft' udloeser hver note en error i den faerdige rapport.
\usepackage[footnote,draft,danish,silent,nomargin]{fixme}
\renewcommand{\thefootnote}{\arabic{footnote}}

%%%% CUSTOM SETTINGS %%%%

% ¤¤ Marginer ¤¤ %
\setlrmarginsandblock{3.5cm}{2.5cm}{*}		% \setlrmarginsandblock{Indbinding}{Kant}{Ratio}
\setulmarginsandblock{2.0cm}{3.0cm}{*}		% \setulmarginsandblock{Top}{Bund}{Ratio}
\checkandfixthelayout 						% Oversaetter vaerdier til brug for andre pakker

%	¤¤ Afsnitsformatering ¤¤ %
\setlength{\parindent}{0mm}           		% Stoerrelse af indryk
\setlength{\parskip}{3mm}          			% Afstand mellem afsnit ved brug af double Enter
\linespread{0.85}								% Linie afstand

% ¤¤ Indholdsfortegnelse ¤¤ %
\setsecnumdepth{subsubsection}		 			% Dybden af nummerede overkrifter (part/chapter/section/subsection)
\maxsecnumdepth{subsubsection}				% Dokumentklassens graense for nummereringsdybde
\settocdepth{subsubsection} 					% Dybden af indholdsfortegnelsen

% ¤¤ Lister ¤¤ %
\setlist{
  topsep=0pt,								% Vertikal afstand mellem tekst og listen
  itemsep=-1ex,								% Vertikal afstand mellem items
} 

% ¤¤ Visuelle referencer ¤¤ %
\usepackage[colorlinks]{hyperref}			% Danner klikbare referencer (hyperlinks) i dokumentet.
\hypersetup{colorlinks = true,				% Opsaetning af farvede hyperlinks (interne links, citeringer og URL)
    linkcolor = black,
    citecolor = black,
    urlcolor = blue
}

% ¤¤ Opsaetning af figur- og tabeltekst ¤¤ %
\captionnamefont{\small\bfseries\itshape}	% Opsaetning af tekstdelen ('Figur' eller 'Tabel')
\captiontitlefont{\small}					% Opsaetning af nummerering
\captiondelim{. }							% Seperator mellem nummerering og figurtekst
\hangcaption								% Venstrejusterer flere-liniers figurtekst under hinanden
\setlength{\belowcaptionskip}{0pt}			% Afstand under figurteksten


\let\oldequation=\equation
\let\endoldequation=\endequation
\renewenvironment{equation}{\vspace{-6mm}\begin{oldequation}}{\end{oldequation}\vspace{-6mm}}

% ¤¤ Navngivning ¤¤ %
\addto\captionsdanish{
	\renewcommand\appendixname{Appendix}
	\renewcommand\contentsname{Table of contest}	
	\renewcommand\appendixpagename{Appendix}
	\renewcommand\appendixtocname{Appendix}
	\renewcommand\cftchaptername{\chaptername~}				% Skriver "Kapitel" foran kapitlerne i indholdsfortegnelsen
	\renewcommand\cftappendixname{\appendixname~}			% Skriver "Appendix" foran appendiks i indholdsfortegnelsen
}

% ¤¤ Kapiteludssende ¤¤ %
\definecolor{numbercolor}{gray}{0.7}		% Definerer en farve til brug til kapiteludseende
\newif\ifchapternonum

\makechapterstyle{jenor}{					% Definerer kapiteludseende frem til ...
  \renewcommand\beforechapskip{0pt}
  \renewcommand\printchaptername{}
  \renewcommand\printchapternum{}
  \renewcommand\printchapternonum{\chapternonumtrue}
  \renewcommand\chaptitlefont{\fontfamily{pbk}\fontseries{db}\fontshape{n}\fontsize{20}{30}\selectfont\raggedleft}
  \renewcommand\chapnumfont{\fontfamily{pbk}\fontseries{m}\fontshape{n}\fontsize{1in}{0in}\selectfont\color{numbercolor}}
  \renewcommand\printchaptertitle[1]{%
    \noindent
    \ifchapternonum
    \begin{tabularx}{\textwidth}{X}
    {\let\\\newline\chaptitlefont ##1\par} 
    \end{tabularx}
    \par\vskip-2.5mm\hrule
    \else
    \begin{tabularx}{\textwidth}{Xl}
    {\parbox[b]{\linewidth}{\chaptitlefont ##1}} & \raisebox{-15pt}{\chapnumfont \thechapter}
    \end{tabularx}
    \par\vskip2mm\hrule
    \fi
  }
}											% ... her

\chapterstyle{jenor}						% Valg af kapiteludseende - Google 'memoir chapter styles' for alternativer

% ¤¤ Sidehoved ¤¤ %

\makepagestyle{AAU}							% Definerer sidehoved og sidefod udseende frem til ...
\makepsmarks{AAU}{%
	\createmark{chapter}{left}{shownumber}{}{. \ }
	\createmark{section}{right}{shownumber}{}{. \ }
	\createplainmark{toc}{both}{\contentsname}
	\createplainmark{lof}{both}{\listfigurename}
	\createplainmark{lot}{both}{\listtablename}
	\createplainmark{bib}{both}{\bibname}
	\createplainmark{index}{both}{\indexname}
	\createplainmark{glossary}{both}{\glossaryname}
}
\nouppercaseheads											% Ingen Caps oenskes


\makeevenhead{AAU}{Gr837}{}{\leftmark}					    % Definerer lige siders sidehoved (\makeevenhead{Navn}{Venstre}{Center}{Hoejre})
\makeoddhead{AAU}{\rightmark}{}{Aalborg University}			% Definerer ulige siders sidehoved (\makeoddhead{Navn}{Venstre}{Center}{Hoejre})
\makeheadrule{AAU}{\textwidth}{0.5pt}						% Tilfoejer en streg under sidehovedets indhold
\makefootrule{AAU}{\textwidth}{0.5pt}{1mm}					% Tilfoejer en streg under sidefodens indhold

\copypagestyle{AAUchap}{AAU}								% Sidehoved for kapitelsider defineres som standardsider, men med blank sidehoved
\makeoddhead{AAUchap}{}{}{}
\makeevenhead{AAUchap}{}{}{}
\makeheadrule{AAUchap}{\textwidth}{0pt}
\aliaspagestyle{chapter}{AAUchap}							% Den ny style vaelges til at gaelde for chapters
															% ... her
															
\pagestyle{AAU}												% Valg af sidehoved og sidefod


%%%% CUSTOM COMMANDS %%%%

% ¤¤ Specielle tegn ¤¤ %
\newcommand{\decC}{^{\circ}\text{C}}
\newcommand{\dec}{^{\circ}}


%%%% ORDDELING %%%%

\hyphenation{}

%%%% GlOSSARIES %%%%
%\usepackage[toc,nowarn]{glossaries}
%\makeglossaries
%
%% \newacronym{label}{short name}{long name}







%% \newglossaryentry{label}{name={the real name}, description={robots are the better humans}}

%\makeatletter\@openrightfalse\makeatother


%%%% COOL EXTRA FEATURES %%%%%
\usepackage{lastpage}
\usepackage{todonotes}
\newcommand{\todotom}[1]{\todo[color=red!40, author=Tom]{#1}}
\graphicspath{{./rapport/billeder/}}
\newcommand{\todoque}[1]{\todo[color=red, author=Question:]{#1}}
\graphicspath{{./rapport/billeder/}}
\usepackage{epstopdf}


%%%% WARNING HACKS %%%%
\hfuzz=\maxdimen
\tolerance=100000
\hbadness=\maxdimen
\vbadness=\maxdimen
\vfuzz=\maxdimen


%%%% NICE LOOKING REFERENCE %%%%
\newcommand{\itref}[1]{\textit{\ref{#1}}}
\newcommand{\chapref}[1]{Chapter \textit{\ref{#1}: \nameref{#1}}}
\newcommand{\secref}[1]{Section \textit{\ref{#1}: \nameref{#1}}}
\newcommand{\figref}[1]{\emph{Figure \ref{#1}}}
\newcommand{\appref}[1]{\emph{Appendix: \ref{#1}}}
\newcommand{\tabref}[1]{\emph{Table: \ref{#1}}}
\newcommand{\coderef}[1]{\emph{Listings: \ref{#1}}}
\renewcommand{\eqref}[1]{\emph{Equation: (\ref{#1})}}


%%%% TIKZ MAGIC %%%%
\usepackage{schemabloc}
\usetikzlibrary{circuits}
\usepackage{tikz}
\usetikzlibrary{shapes,arrows}
\usepackage{pgfplots}
\usetikzlibrary{plotmarks}
\usetikzlibrary{fit}
\pgfplotsset{compat=newest}
\pgfplotsset{filter discard warning=false}
%\usepackage[americanresistors,americaninductors,american voltages]{circuitikz}
\usetikzlibrary{calc}
\usetikzlibrary{patterns}
\ctikzset{bipoles/thickness=1}
\ctikzset{bipoles/length=0.8cm}
\ctikzset{bipoles/diode/height=.375}
\ctikzset{bipoles/diode/width=.3}
\ctikzset{tripoles/thyristor/height=.8}
\ctikzset{tripoles/thyristor/width=1}
\ctikzset{bipoles/vsourceam/height/.initial=.7}
\ctikzset{bipoles/vsourceam/width/.initial=.7}
\tikzstyle{every node}=[font=\small]
\tikzstyle{every path}=[line width=0.8pt,line cap=round,line join=round]
\definecolor{AleeRed}{rgb}{0.5,0,0}
\newcommand\addvmargin[1]{
  \node[fit=(current bounding box),inner ysep=#1,inner xsep=0]{};
}
\usetikzlibrary{positioning}

% \usetikzlibrary{external}
% \tikzexternalize[prefix=tikz/]

%%%% CDO FIX %%%%
\usepackage{silence}
\WarningsOff[pgfplots]
\WarningFilter{latex}{Marginpar on page}
\WarningFilter{memoir}{You are using the caption package with the memoir class}
\WarningsOff[latex]
\WarningFilter{LaTeX Font Warning:}{Font shape `U/stmry/b/n' undefined(Font)              using `U/stmry/m/n' instead on input line 5.}